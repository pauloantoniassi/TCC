\selectlanguage{brazil}
\begin{abstract}
    
    As análises de desempenho são ferramentas exigidas em todo processo produtivo de uma empresa que visa economia, efetividade, eficiência e a consequente satisfação do cliente com o serviço prestado. Neste contexto tem destaque privilegiado o desenvolvimento, acompanhamento e investimento em software. Os métodos atualmente apresentados na doutrina relacionada, leva em consideração a quantidade de trabalho e o tempo dispendido para a sua realização. No entanto, não obstante tratar-se de método de boa qualidade, uma análise pormenorizada revela que a comparação direta entre dois programadores fica prejudicada, na medida em que não leva em conta que os programadores nem sempre estão disponíveis exclusivamente para o mesmo trabalho, bem como não observa que nem sempre possuem as mesmas disponibilidades e outras atribuições, normalmente cumulativas com as de programadores, tais como gestão de equipe e atividades de suporte ao cliente. Esta pesquisa visa propor um novo método de cálculo de desempenho, buscando normalizar as diferentes variáveis entre os programadores. O que se apresenta neste estudo é a possibilidade que permite a comparação da entrega bruta entre os membros da equipe de trabalho sem cálculos adicionais.
\end{abstract}

\selectlanguage{english} 
\begin{abstract}
   Performance analyses are tools required in every productive process of companies that aim at economy, effectiveness, efficiency and the consequent satisfaction of the client with the service provided. In this context, the development, monitoring and investment in software have been highlighted. The methods currently presented in the related literature take into consideration the amount of work and the time spent for its realization. However, despite being a good quality method, a detailed analysis reveals that the direct comparison between two programmers is impaired, since it does not take into account that programmers are not always available exclusively for the same work, as well as does not observe that they do not always have the same availability and other assignments, usually cumulative with those of programmers, such as team management and customer support activities. This research aims to propose a new method of performance calculation, seeking to normalize the different variables among programmers. What is presented in this study is the possibility that allows the comparison of the gross performance among the members of the work team without additional calculations.
\end{abstract}

\selectlanguage{brazil}