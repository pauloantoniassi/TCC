\section{Introdução}
Os métodos ágeis surgiram na década passada propondo uma abordagem diferente para o desenvolvimento de software \cite{sato2007usoMetricas}. Segundo o estudo descrito por  \citeauthor{sato2007usoMetricas}, a metodologia ágil possui melhor adaptação às alterações de mercado e escopo de projeto, reduzindo custos e entregando maior valor ao projeto e consequentemente ao cliente.\par

Dentro das propostas da metodologia ágil, muitas empresas utilizam técnicas de rastreamento de desempenho e produção, pautadas principalmente na medida de tempo e quantidade de trabalho, em conformidade com o método citado por \citeonline{barbaran1998indicadores}. Esta técnica expõe algumas formas de mensuração do desempenho já conhecidas no mercado, que levam sempre em consideração equipes dedicadas ao desenvolvimento de software.\par 

Um dos métodos apresentados por \citeonline{barbaran1998indicadores} é calculado levando em consideração Pontos por Função e Pessoas/Tempo disponíveis por mês, considerando que todos os desenvolvedores possuem a mesma disponibilidade de tempo no período.

No entanto, com base em pesquisas de campo, notamos que este método pode não ser tão eficaz quando nos deparamos com algumas situações muito corriqueiras na empresas. Tais situações podem ocorrer quando os membros da equipe de trabalho não são exclusivamente dedicados ao desenvolvimento, ou realizam outras tarefas, além daquelas mensuráveis de desenvolvimento, a exemplo, tarefas com suporte ao cliente, ou até mesmo o exercício de gestão de equipe.\par

Tendo este cenário em mente e usando para análise dados de uma empresa voluntária, este estudo busca propor um método de cálculo de desempenho que leve em consideração também o tempo dispendido pelo desenvolvedor em tarefas não mensuráveis, equiparando as medidas de desempenho às tarefas mensuráveis.\par

Desta forma, é necessário entender como é realizada a medição de desempenho da equipe na fábrica de software da empresa, neste caso, a empresa voluntária participante deste estudo. A partir daí, determinar quais variáveis podem ser utilizadas para realizar as medições e acompanhamentos na fábrica. Detalhar as técnicas que poderão ser utilizadas para automatizar o processo de coleta de dados e por fim, construir um modelo de cálculo de desempenho e validar os dados in loco, comparando com o modelo de cálculo já utilizado na empresa selecionada.