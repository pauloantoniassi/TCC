\section{Referencial Teórico}

\subsection{ Importância da Avaliação de Desempenho}
Ao se tratar de métricas de desempenho é necessário citar alguns dos objetivos de se ter este tópico em análise. Alguns dos pontos citados por \citeonline{bruno2019GPDesempenho} como objetivos para a avaliação de desempenho é a gestão de equipes e metas, como por exemplo investimento em treinamentos, transferências internas, aperfeiçoamentos, promoções, reajuste salarial ou até mesmo demissões. \citeonline{qulturerocksdesempenho} apresenta como consequência deste trabalho de gestão de equipes e metas, a melhoria na tomada de decisões e maior retorno para a empresa, tais como eficiência na prestação do serviço e otimização dos trabalhos.\par
Na área de desenvolvimento de software, \citeonline{barbaran1998indicadores} demonstram um dos métodos utilizados no cálculo do desempenho da equipe de desenvolvedores. Este método leva em consideração Pontos por Função e Pessoas disponíveis por mês, considerando-se que todos os desenvolvedores possuem a mesma disponibilidade de tempo no período. Estes autores salientam que esta técnica, ou suas variações, são uns dos principais métodos de cálculo de desempenho utilizados atualmente.

\subsection{Scrum e Sprints}
Segundo \citeonline{scrumguide}, Scrum é um framework para desenvolvimento e manutenção de projetos complexos, que começou a ser utilizado no início dos anos 90. A Scrum tem sua origem no jogo de rugby, praticado na Inglaterra, onde os jogadores se reuniam para "resolver um problema e refletir sobre suas conquistas e fracassos para melhorar sempre" \cite{scrumatlassian}. Esta ideia de refletir e aprender com o passado para aprimorar questões futuras, foi levada para o desenvolvimento ágil e rapidamente se tornou conhecida na engenharia de software.\par 

Embora, em sua definição original, Scrum não seja uma metodologia \cite{scrumguide}, atualmente o Scrum é considerado uma das principais metodologias ágeis para a gestão e planejamento do desenvolvimento de software. O Scrum define o desenvolvimento em pequenos ciclos, podendo ser semanais ou até mensais. Desta forma, permite gerenciar mais facilmente as possíveis alterações no escopo do projeto fim. Para um ciclo é dado o nome de Sprint \cite{scrum}.\par

Sprint é o intervalo de tempo dentro do qual as atividades planejadas devem ser executadas. Normalmente uma Sprint possui entre 1 semana e 1 mês.
A lista de tarefas planejadas a serem executadas, recebe o nome de Sprint Backlog. O Sprint Backlog deve ser construído a cada nova Sprint, tendo sempre em mente o tempo disponível, o tempo necessário e a complexidade de cada tarefa. Ao se imaginar a regularidade dos acontecimentos, sem intempéries, ao final da sprint todas as atividades planejadas para a equipe de desenvolvimento devem estar concluídas, ou seja, o Sprint Backlog deve estar sem pendências. Nestas condições, o escopo de trabalho se mantém pequeno e facilmente gerenciável.

\subsection{Equipes Multitarefas}
Segundo o dicionário \citeonline{diciomultitarefas}, Multitarefas é a "capacidade para realizar mais de uma tarefa ao mesmo tempo". No contexto deste estudo, uma equipe multitarefas é caracterizada pela atribuição de papeis aos integrantes de múltiplas funções, dentre elas as do desenvolvimento de software, como por exemplo, suporte ao cliente e até mesmo a gestão de equipe. É de se destacar que tanto a gestão de equipe quanto o suporte ao cliente, podem consumir o tempo que o desenvolvedor teria disponível para o exercício da tarefa principal, qual seja, o desenvolvimento da tarefa especificamente  que lhe foi atribuída. \par

Analisada a questão tempo dispendido, exercício de multitarefas e o custo benefício para a empresa, pode-se afirmar que há um considerável prejuízo que, na maioria das vezes, não recebe a atenção merecida para análise profunda de custo e benefício. Em síntese, de acordo com \apudonline{FIPE2017}{rabello2017gestao}, o salário médio de um analista é R\$ 4.488,00 reais para o desempenho de 200 horas mensais. Conforme \citeonline{rabello2017gestao}, que cita um estudo da Associação Americana de Psicologia, um profissional que recebe multitarefas, pode ter em média 40\% do seu tempo é desperdiçado na alternância da tarefa principal para a tarefa segundaria. Tendo isto em mente, é possível verificar um prejuízo potencial de mais de R\$ 21.000,00 mil reais por funcionário por ano. Situação que deve ser levada em conta por qualquer empresa que busque economia, celeridade, satisfação do cliente e principalmente motivação dos trabalhadores envolvidos no desenvolvimento da tarefa.

\subsection{Pontos da Tarefa}
De acordo com os ensinamento de \citeonline{scrumorgstorypoints}, pontos da história (User Story Points) é uma unidade de medida genérica, utilizada para mensurar uma determinada atividade. Segundo \citeonline{scrumorgstorypoints}, User Story Points são essenciais para analisarmos a velocidade da equipe, uma vez que tem o papel de manter o equilíbrio do tempo gasto em diferentes tarefas atribuídas não deixando de considerar a complexidade de cada tarefa.\par
\citeonline{storypointsculturaagilexemplo} traz como exemplo de Story Points uma caminhada entre dois amigos. O primeiro deles, já acostumado a prática da corrida, faz o percurso de 7 quilômetros em 35 minutos. O segundo, alheio a prática da atividade física, portanto despreparado, para desenvolver o mesmo percurso, exigirá o tempo de 50 minutos. Neste exemplo, o denominador comum é a distância que ambos concordaram em percorrer, ou seja, 7 quilômetros. Nestas condições, a distância faz o papel do story point, ou seja a distância, ou a complexidade da tarefa, será sempre constante, independente do tempo consumido para a sua realização.

\subsection{Modelo Inicial}
A forma de cálculo de desempenho apresentada por \citeonline{barbaran1998indicadores} é demonstrada abaixo:

\bigskip

\begin{equation}
    DES = \frac{PF}{PES}
\end{equation}

\begin{itemize}
    \item PF: Soma dos pontos das tarefas implementadas
    \item PES: Número de pessoas disponíveis por mês
    \item DES: Desempenho calculado
\end{itemize}

\bigskip

Já o modelo inicialmente utilizada pela empresa participante do estudo é uma variação do modelo original citado por \citeonline{barbaran1998indicadores} em seu trabalho. Nesta variação do modelo, utiliza-se as variáveis abaixo, podendo estar no escopo da equipe ou do desenvolvedor:
\bigskip

\begin{equation}
    DES = \frac{PA}{HR}
\end{equation}

\begin{itemize}
    \item PA: Soma dos pontos das tarefas implementadas e aceitas nos testes. Tarefas implementadas porém não aprovadas em teste não são consideradas.
    \item HR: Soma das horas gastas na implementação de stories e correção de bugs, mesmo que estes não possuam pontos de tarefa.
    \item DES: Desempenho calculado
\end{itemize}
\bigskip