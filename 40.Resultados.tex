\section{Resultados}

Para analisarmos os resultados, utilizamos uma amostra de 4 Sprints da empresa participante, equivalente a aproximadamente 4 semanas de trabalho.\par
Com o método original utilizado pela empresa em estudo temos a seguinte tabela de desempenho, abaixo indicada, agrupando o desenvolvedor como single task, que atua exclusivamente com tarefas mensuráveis ou multi task, que atuam em outras tarefas além das mensuráveis.

\begin{center}
\begin{tabular}{ | l | l | r | r | r | } 
 \hline
 Programador & Tipo & Pontos & Horas & Razão de \\ 
 & & Realizados & Utilizadas & Desempenho \\ 
 & & & & (pontos/hora) \\ \hline
 Dev 1 & Single Task & 99  & 414 & 0,24 \\
 Dev 2 & Single Task & 69  & 367 & 0,19 \\
 Dev 3 & Single Task & 76  & 351 & 0,22 \\
 Dev 4 & Multi Task  & 66  & 247 & 0,27 \\
 Dev 5 & Single Task & 85  & 383 & 0,22 \\
 Dev 6 & Multi Task  & 105 & 419 & 0,25 \\
 \hline
\end{tabular}
\end{center}

Conforme já citado neste trabalho, a empresa utiliza uma variação do cálculo de desempenho citado por \citeonline{barbaran1998indicadores}, que é definido pela seguinte fórmula:\par

\begin{equation}
    DESEMPENHO = { PONTOS\_ACEITOS \over HORAS\_UTILIZADAS }
\end{equation}

Utilizando a fórmula proposta neste estudo é possível elaborar a tabela abaixo, com o mesmo agrupamento da tabela anterior, ou seja single ou multi task.

\begin{center}
\begin{tabular}{ | l | l | r | r | r | } 
 \hline
 Programador & Tipo & Pontos & Horas & Razão de \\ 
 & & Realizados & Utilizadas & Desempenho \\ 
 & & & & (pontos/hora) \\ \hline
 Dev 1 & Single Task & 169,2 & 453,2 & 0,37 \\
 Dev 2 & Single Task & 98,0  & 407,1 & 0,24 \\
 Dev 3 & Single Task & 105,1 & 381,3 & 0,27 \\
 Dev 4 & Multi Task  & 145,7 & 384,8 & 0,37 \\
 Dev 5 & Single Task & 117,4 & 415,7 & 0,28 \\
 Dev 6 & Multi Task  & 194,4 & 445,9 & 0,43  \\
 \hline
\end{tabular}
\end{center}

As duas tabelas, embora parecidas, pedem interpretações diferentes. A primeira tabela traz apenas os dados mensuráveis, os que têm pontos de história, desconsiderando as demais dados de tarefas não mensuráveis. A interpretação desta tabela deve levar em consideração, numa análise empírica e manual, que a entrega pode ter sido afetada por diversos fatores que não foram considerados no cálculo.\par
Já a segunda tabela traz o método proposto por este trabalho, que já inclui outras tarefas que consumiram tempo do programador, tornando a análise mais simples e rápida para a equipe de gestão. \par 
É possível observar que a entrega bruta dos desenvolvedores Multi Task ficou levemente superior aos demais integrantes da equipe, diferentemente do que se esperava durante a preparação do estudo. Com o resultado obtido é possível proceder a análises das causas que impactaram na entrega das tarefas. Algumas hipóteses, embora não sendo objeto deste estudo, foram levantadas, entre elas está a distribuição de tarefas irregular entre os integrantes, bem como diferentes níveis de expertises exigidas.\par

Podemos evidenciar a diferença entre o modelo anteriormente utilizado pela empresa e o novo modelo ora apresentado, procedendo-se a uma comparação simples entre as duas tabelas:

\begin{center}
\begin{tabular}{ | l | l | r | r | r | } 
 \hline
 Programador & Tipo & Razão de & Razão de & Diferença \\ 
 & & Desempenho & Desempenho & Bruta e  \\ 
 & & Mod. Original & Mod. Proposto & Percentual \\ \hline
 Dev 1 & Single & 0,24 & 0,37 & 0,13 / 54\% \\
 Dev 2 & Single & 0,19 & 0,24 & 0,05 / 26\% \\
 Dev 3 & Single & 0,22 & 0,27 & 0,05 / 23\% \\
 Dev 4 & Multi  & 0,27 & 0,37 & 0,10 / 37\% \\
 Dev 5 & Single & 0,22 & 0,28 & 0,06 / 27\% \\
 Dev 6 & Multi  & 0,25 & 0,43 & 0,18 / 72\% \\
 \hline
\end{tabular}
\end{center}

Observando-se a variação entre os dois modelos de métodos, os resultados indicam uma alteração média de 39,8\% na razão de desempenho. Nesta tabela, a diferença percentual indica o quão significativo é as tarefas não mensuráveis no dia a dia do colaborador.\par
Com auxilio destas tabelas foi possível realizar a validação do modelo desenvolvido in loco, com a participação da gestão da empresa, que aprovou o modelo proposto e iniciou a sua implementação.