\section{Resultados}

Com o método original utilizado pela empresa em estudo temos a seguinte tabela de desempenho, abaixo indicada, agrupando o desenvolvedor como single task, que atua exclusivamente com tarefas mensuráveis ou multi task, que atuam em outras tarefas além das mensuráveis.

\begin{center}
\begin{tabular}{ | l | l | r | r | r | } 
 \hline
 Programador & Tipo & Pontos & Pontos & Razão de \\ 
 & & Desejáveis & Realizados & Desempenho \\ \hline
 Dev 1 & Single Task & 208 & 99,5 & 0,48 \\
 Dev 2 & Single Task & 208 & 69 & 0,33 \\
 Dev 3 & Single Task & 208 & 76,5 & 0,37 \\
 Dev 4 & Multi Task & 150 & 66 & 0,44 \\
 Dev 5 & Single Task & 208 & 85 & 0,41 \\
 Dev 6 & Multi Task & 208 & 105 & 0,50 \\
 \hline
\end{tabular}
\end{center}

A empresa utiliza o cálculo de desempenho citado por \citeonline{barbaran1998indicadores}, que é definido pela seguinte fórmula:\par
\[
DESEMPENHO = PONTOS\_ACEITOS/PONTOS\_PLANEJADOS
\]

Utilizando a fórmula proposta neste estudo é possível elaborar a tabela abaixo, com o mesmo agrupamento da tabela anterior, ou seja single ou multi task.

\begin{center}
\begin{tabular}{ | l | l | r | r | r | } 
 \hline
 Programador & Tipo & Pontos & Pontos & Razão de \\ 
 & & Desejáveis & Realizados & Desempenho \\ \hline
 Dev 1 & Single Task & 208 & 169,2 & 0,81 \\
 Dev 2 & Single Task & 208 & 98.0 & 0,47 \\
 Dev 3 & Single Task & 208 & 105,1 & 0,51 \\
 Dev 4 & Multi Task & 206 & 145,7 & 0,71 \\
 Dev 5 & Single Task & 208 & 117,4 & 0,56 \\
 Dev 6 & Multi Task & 208 & 194,4 & 0,93 \\
 \hline
\end{tabular}
\end{center}

As duas tabelas, embora parecidas, pedem interpretações diferentes. A primeira tabela traz apenas os dados mensuráveis, os que têm pontos de história, desconsiderando as demais dados de tarefas não mensuráveis. A interpretação desta tabela deve levar em consideração, numa análise empírica e manual, que a entrega pode ter sido afetada por diversos fatores que não foram considerados no cálculo.\par
Já a segunda tabela traz o método proposto por este trabalho, que já inclui outras tarefas que consumiram tempo do programador, tornando a análise mais simples e rápida para a equipe de gestão. \par 
É possível observar que a entrega bruta dos desenvolvedores Multi Task ficou levemente superior aos demais integrantes da equipe, diferentemente do que se esperava durante a preparação do estudo. Com o resultado obtido é possível proceder a análises das causas que impactaram na entrega das tarefas. Algumas hipóteses, embora não sendo objeto deste estudo, foram levantadas, entre elas está a distribuição de tarefas irregular entre os integrantes, bem como diferentes níveis de expertises exigidas.\par

Podemos evidenciar a diferença entre o modelo anteriormente utilizado pela empresa e o novo modelo ora apresentado, procedendo-se a uma comparação simples entre as duas tabelas:

\begin{center}
\begin{tabular}{ | l | l | r | r | r | } 
 \hline
 Programador & Tipo & Razão de & Razão de & Diferença \\ 
 & & Desempenho & Desempenho & Bruta e  \\ 
 & & Mod. Original & Mod. Proposto & Percentual \\ \hline
 Dev 1 & Single Task & 0,48 & 0,81 & 0,33 / 69\% \\
 Dev 2 & Single Task & 0,33 & 0,47 & 0,14 / 42\% \\
 Dev 3 & Single Task & 0,37 & 0,51 & 0,14 / 38\% \\
 Dev 4 & Multi Task & 0,44 & 0,71 & 0,27 / 61\% \\
 Dev 5 & Single Task & 0,41 & 0,56 & 0,15 / 37\% \\
 Dev 6 & Multi Task & 0,50 & 0,93 & 0,43 / 86\% \\
 \hline
\end{tabular}
\end{center}

Observando-se a variação entre os dois modelos de métodos, os resultados indicam uma alteração média de 55,5\% na razão de desempenho, sendo superior a 60\% nos colaboradores MultiTask. Com auxilio destas tabelas foi possível realizar a validação do modelo desenvolvido in loco, com a participação da gestão da empresa, que aprovou o modelo proposto e iniciou a sua implementação.


%Este resultado levou a empresa participante do estudo a iniciar uma nova análise para entender o motivo do estudo ter apontado dados diferentes dos esperados.

%
%
%
%Embora podemos ver que as tarefas atribuídas foram semelhantes, levantamos o questionamento: Aqueles que são "multi task" estão se esforçando mais para entregar o mesmo número de tarefas mensuráveis que os demais single task mesmo com as demandas não mensuráveis, ou aqueles single task estão "acomodados" usando a taxa de entrega dos multi task em suas tarefas mensuráveis como referência de entrega, ignorando o fato destes terem as tarefas não mensuráveis também?\par