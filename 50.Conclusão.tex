\section{Conclusão}
Com base no estudo de caso de uma empresa da região de Maringá, demonstrou-se uma alternativa viável ao método de cálculo de desempenho tradicional, utilizado na maioria das empresas de software. Para isso, foi necessária a realização de parceria com o departamento pessoal desta empresa e levantamento de dados técnicos, além observação dos trabalhos in loco.\par

Proposto e aplicado o novo modelo, constatou-se resultados satisfatórios para a empresa, que imprimiu maior celeridade e efetividade na elaboração e entrega de suas tarefas. Com este modelo a análise de uma equipe mista tornou-se simples, permitindo comparação direta entre os participantes. Apoiado no novo método, o departamento pessoal da empresa participante implementou um critério de premiação por desempenho na equipe (gamificação/ludificação) que utiliza o desempenho bruto como métrica de bonificação.\par


\subsection{Trabalhos Futuros}
Podemos observar, através dos resultados obtidos neste estudo, que, diferentemente do esperado, as métricas indicam que os desenvolvedores multitask possuem desempenho ligeiramente superior em relação ao demais desenvolvedores da equipe.\par

Para o futuro, propõe-se uma análise empírica da equipe, para compreender o que leva o desenvolvedor multi-task a entregar mais pontos quando comparado aos programadores single-task da mesma equipe, mesmo outros estudos indicando que o multi-tasking reduz o desempenho de um funcionário.\par 

Propõe-se o seguinte questionamento: "Como a equipe vê e percebe o desempenho de um desenvolvedor multitask no tocante às tarefas não-mensuráveis e como essa percepção afeta o desempenho dos integrantes single-task?".